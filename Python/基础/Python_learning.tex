%-*- coding: UTF-8 -*-
\documentclass[UTF8]{ctexart}
\usepackage{geometry}
\geometry{a4paper,centering,scale=0.8}
\usepackage{graphicx}
\usepackage{amsmath}
\usepackage{textcomp}
\usepackage{amsthm}
\usepackage{amssymb}
\usepackage{float}
\usepackage{minted}

\title{\heiti Python 基础篇}
\author{卢婧宇}

\begin{document}
\maketitle
\tableofcontents

\newpage
\section{开胃菜}
虽然 Python 易于使用,但它却是一门完整的编程语言;与 Shell 脚本或批处理文件相比,它为编写大型程序提供了更多的结构和支持。
另一方面,Python提供了比C更多的错误检查,并且作为一门高级语言,它内置支持高级的数据结构类型,例如:灵活的数组和字典。

Python允许你将程序分割为不同的模块,以便在其他的Python程序中重用。Python内置提供了大量的标准模块,你可以将其用作程序的基础,
或者作为学习Python编程的示例。这些模块提供了诸如文件I/O、系统调用、Socket支持,甚至类似Tk的用户图形界面(GUI)工具包借口。

Python是一门解释型语言,因为无需编译和链接,你可以在程序开发中节省宝贵的时间。Python解释器可以交互的使用,这使得试验语言的特性、
编写临时程序或在自底向上的程序开发中测试方法非常容易。你甚至还可以把它当做一个桌面计算器。

Python让程序编写的紧凑和可读。用Python编写的程序通常比同样的C、C++或Java程序更短小,原因如下:
\begin{itemize}
  \item 高级数据结构使你可以在一条语句中表达复杂的操作
  \item 语句组使用缩进代替开始和结束大括号来组织
  \item 变量或参数无需声明
\end{itemize}

Python是可扩展的:如果你会C语言编程便可以轻易地为解释器\footnote{解释器,又译为直译器,是一种电脑程序,能够把高级编程语言一行一行直接转译运行。}
添加内置函数或模块,或者为了对性能瓶颈作优化,或者将Python程序与只有二进制形式的库(比如末各专业的商业图形库)连接起来。
一旦你真正掌握了它,你可以将Python解释器集成进某个C应用程序,并把它当作那个程序的扩展命令行语言。
\section{使用Python解释器}
\subsection{调用Python解释器}
Python解释器通常被安装在目标机器的 $/$usr$/$local$/$bin$/$python3.5 目录下。将 $/$usr$/$local$/$bin 目录包含进 Unix shell的搜索路径里,
以确保可以通过输入:Python3.5 命令启动。由于Python解释器的安装路径是可选的,这也可能是其他路径,你可以联系安装Python的用户或系统管理员确认
(例如,/usr/local/python 就是一个常见的选择)。文件结束符Unix系统是:Control + D

第二种启动Python解释器的方法是 Python -c command [arg] ...,这种方法可以在命令行执行Python语句,类似于shell中的 -c选项。由于Python语句通常会包含
空格或其他特殊shell字符,一般建议将命令用单引号包裹起来。

有一些Python模块也可以当做脚本使用。你可以使用 python -m module [arg] ... 命令调用它们,这类似在命令行中键入完整的路径名执行模块源文件一样。
使用脚本文件时,经常会运行脚本然后进入交互模式。这也可以通过在脚本之前加上 -i参数来实现。
\subsubsection{交互模式}
从tty读取命令时,我们称解释器工作于交互模式。这种模式下他根据主提示符来执行,主提示符通常标识为三个大于号($>>>$);继续的部分被称为从属提示符,由三个
点标识(...)。在第一行之前,解释器打印欢迎信息、版本号和授权提示
\subsubsection{参数传递}
调用解释器时,脚本名和附加参数传入一个名为 sys.argv的字符串列表。你能够获取这个列表通过执行 import sys,列表的长度大于等于1;
没有给定脚本和参数时,它至少也有一个元素:sys.argv[0]此时为空字符串。脚本名指定为`-'(表示标准输入)时,sys.argv[0]被定义为`-',使用 -c 指令时,sys.argv[0]
被设定为`-c'。使用-m模块参数时,sys.argv[0]被设定为指定模块的全名。-c 指令或者 -m 模块之后的参数不会被Python解释器的选项处理机制所截获,而是留在sys.argv中,
供脚本命令操作。
\section{Python简介}
输入和输出分别由大括号和句号提示符($>>>$ 和 ...)标注。

Python中的注释以\#字符起始,直至实际的行尾。注释可以从行首开始,也可以在空白或代码之后,但是不出现在字符串中。
文本字符串中的\#字符仅仅表示\#。
\subsection{将Python当做计算器}
\subsubsection{数字}
解释器表现得就像一个简单的计算器:可以向其录入一些表达式,它会给出返回值。表达式语法很直白:运算符$+,-,*$和$/$与其他语言一样;括号用于分组。

整数(例如,2,4,20)的类型是int,带有小数部分的数字(例如,5.0,1.6)的类型是float。Python2除法想要得到浮点型,除数或被除数应是浮点型。
Python3除法($/$)永远返回一个浮点数,如果要使用floor除法并且得到整数结果(丢掉任何小数部分),可以使用($//$)运算符;要计算余数使用 \% 。
可以使用$**$运算符计算幂乘方。用等号($=$)用于给变量赋值。赋值之后在下一个提示符之前不会有任何结果显示。

变量在使用前必须“定义”(赋值),否则会出错。浮点数有完整的支持;整数和浮点数的混合计算中,整数会被转换为浮点数。

交互模式中,最近一个表达式的值赋给变量`\_'。这样我们就可以吧它当做一个桌面计算器:

\begin{minted}{python}
>>> t=12.5/100
>>> p=100.50
>>> p*t
12.5625
>>> p+ _
113.0625
\end{minted}

此变量对于用户是只读的。不要尝试给它赋值——你只会创建一个独立的同名局部变量,它屏蔽了系统内置变量的魔术效果。

除了int和float,Python还会支持其它数字类型,例如Decimal和Fraction。Python还内建支持复数,使用后缀 j 或 J 表示虚数部分
(例如,3+5j)。
\subsubsection{字符串}
相比数值,Python也提供了可以通过几种不同方式表示的字符串。它们可以用单引号(`...')或双引号(``...'')标识。
 \textbackslash 可以用来转义引号:
\begin{minted}{python}
>>> 'spam eggs' #single quotes
'spam eggs'
>>>'doesn\'t'  #use \' to escape the single quote...
"doesn't"
>>> '"Yes,"he said.'
'"Yes,"he said.'
>>> "\"Yes,\"he said."
'"Yes,"he said.'
>>> '"Isn\'t,"she said.'
'"Isn\'t,"she said.'
\end{minted}

在交互式解释其中,输出的字符串会用引号引起来,特殊字符会用反斜杠转义。虽然可能和输入看上去不太一样,但是两个字符串是相等的。
如果字符串中只有单引号而没有双引号,就用双引号引用,否则用单引号引用。
print()函数生成可读性更好的输出,它会省去引号并且打印出转义后的特殊字符:
\begin{minted}{python}
>>> '"Isn\'t,"she said.'
'"Isn\'t,"she said.'
>>> print('"Isn\'t." she said.')
"Isn't." she said.
>>> s='First line.\nSecond line.'  # \n means newline
>>> s  # without print(), \n is included in the output
'First line.\nSecond line.'
>>> print(s)  #with print(),\n produces a new line
First line.
Second line.
\end{minted}

如果你前面带有 \textbackslash 的字符被当作特殊字符,你可以使用原始字符串,方法是在第一个引号前面加上一个r。

字符串文本能够分成多行。一种方法是使用三引号:\verb|"""|...\verb|"""|或者 ```...'''。行尾换行符会自动包含到字符串中,
但是可以在行尾加上 \textbackslash
来避免这个行为。可以使用反斜杠为行结尾的字符串,他表示下一行在逻辑上是本行的后续内容。

字符串可以由 $+$ 操作符连接(粘到一起),可以由 $*$ 表示重复。相邻两个字符串文本自动连接在一起。
它只用于两个字符串文本,不能用于字符串表达式:
\begin{minted}{python}
>>> prefix='Py'
>>> prefix 'thon' # can't concatenate a variable and a string literal
...
SyntaxError: invalid syntax
>>> ('un'*3) 'ium'
...
SyntaxError: invalid syntax
\end{minted}


如果你想连接多个变量或者连接一个变量和一个字符串文本,使用 $+$。这个功能在你想切分很长的字符串的时候特别有用:
\begin{minted}{python}
>>> text=('Put several string within parerheses'
...       'to have them joined together.')
>>> text
'Put several string within parerhesesto have them joined together.'
\end{minted}

字符串也可以被截取(检索)。类似于C,字符串的第一个字符索引为0。Python没有单独的字符类型;一个字符就是一个简单的长度为1的字符串。
\begin{minted}{python}
>>> word='Python'
>>> word[0] # character in position 0
'P'
>>> word[5] # character in position 5
'n'
\end{minted}
索引也可以是负数,这将导致从右边开始计算。例如:
\begin{minted}{python}
>>> word[-1] #last character
'n'
>>> word[-2] #second-last character
'o'
>>> word[-6]
'P'
\end{minted}

请注意$-0$实际上就是0,所以他不会导致从右边开始计算。

除了索引,还支持切片。索引用于获得单个字符,切片让你获得一个子字符串:
\begin{minted}{python}
>>> word[0:2] # character from position 0(included)to 2 (excluded)
'Py'
>>> word[2:5] # character from position 2(included) to 5(excluded)
'tho'
\end{minted}

注意,包含起始的字符,不包含末尾的字符。这使得s[:i]+s[i:]永远等于s,:
\begin{minted}{python}
>>> word[:2]+word[2:]
'Python'
>>> word[:4]+word[4:]
'Python'
\end{minted}

切片的索引有非常有用的默认值;省略的第一个索引默认为零,省略的第二个索引默认为切片的字符串的大小。:
\begin{minted}{python}
>>> word[:2]
'Py'
>>> word[4:]
'on'
>>> word[-2:]
'on'
\end{minted}

有个办法可以很容易地记住切片的工作方式:切片时的索引是在两个字符之间。左边第一个字符的索引为0,而长度为n的字符串最后一个字符的右界索引为n。例如:

+---+---+---+---+---+---+

| P | y | t | h | o | n |

+---+---+---+---+---+---+

0   1   2   3   4   5   6

-6  -5  -4  -3  -2  -1

文本中的第一行数字给出字符串的索引点0...6。第二行给出相应的负索引。切片是从i到j两个数值表示的边界之间的所有字符。

对于非负索引,如果上下都在边界内,切片长度就是两个索引之差。

试图使用太大索引会导致错误。
\begin{minted}{python}
>>> word[42] # the word only has 6 characters
Traceback (most recent call last):
  File "<stdin>", line 1, in <module>
IndexError: string index out of range
\end{minted}

Python能够优雅地处理那些没有意义的切片索引:一个过大的索引值(即下标值大于字符串实际长度)将被字符串实际长度所替代,
当上边界比下边界大时(即切片左值大于右值)就返回空字符串:
\begin{minted}{python}
>>> word[4:42]
'on'
>>> word[42:]
''
\end{minted}

Python字符串不可以被更改——它们是不可变的。因此,赋值给字符串索引的位置会导致错误。:
\begin{minted}{python}
>>> word[0]='J'
...
TypeError: 'str' object does not support item assignment
>>> word[2:]='py'
...
TypeError: 'str' object does not support item assignment
\end{minted}

如果你需要一个不同的字符串,你应该创建一个新的:
\begin{minted}{python}
>>> 'J'+word[1:]
'Jython'
>>> word[:2]+'py'
'Pypy'
\end{minted}

内置函数 len() 返回字符串长度:
\begin{minted}{python}
>>> s='supercalifragilisticexpialidocious'
>>> len(s)
34
\end{minted}

\subsubsection{列表}
Python有几个复合数据类型,用于表示其它的值。最通用的是list(列表),它可以写作中括号之间的一列逗号分隔的值。列标的元素不必是同一类型:
\begin{minted}{python}
>>> squares=[1, 4, 9, 16, 25]
>>> squares
[1, 4, 9, 16, 25]
\end{minted}

就像字符串(以及其它所有内建的序列类型)一样,列表可以被索引和切片:
\begin{minted}{python}
>>> squares[0] # indexing returns the item
1
>>> squares[-1]
25
>>> squares[-3:] #slicing return a new list
[9, 16, 25]
\end{minted}

所有的切片操作都会返回一个包含请求的元素的新列表。这意味着下面的切片操作返回列表一个新的(浅)拷贝副本:
\begin{minted}{python}
>>> squares[:]
[1, 4, 9, 16, 25]
\end{minted}

列表也支持连接这样的操作:
\begin{minted}{python}
>>> squares + [36, 49, 64, 81, 100]
[1, 4, 9, 16, 25, 36, 49, 64, 81, 100]
\end{minted}

不像不可变的字符串,列表是可变的,它允许修改元素:
\begin{minted}{python}
>>> cubes=[1, 8, 27, 65, 125] #something's wrong here
>>> 4 ** 3 #the cub of 4 is 64,not 65!
64
>>> cubes[3]=64 # replace the wrong value
>>> cubes
[1, 8, 27, 64, 125]
\end{minted}

还可以使用 append()方法在列表的末尾添加新的元素:
\begin{minted}{python}
>>> cubes.append(216) #add the cube of 6
>>> cubes.append(7 ** 3) # add the cube of 7
>>> cubes
[1, 8, 27, 64, 125, 216, 343]
\end{minted}

也可以对切片赋值,此操作可以改变列表的尺寸,或清空它:
\begin{minted}{python}
  >>> letters=['a','b','c','d','e','f','g']
  >>> letters
  ['a', 'b', 'c', 'd', 'e', 'f', 'g']
  >>> #replace some values
  ... letters[2:5]=['C','D','E']
  >>> letters
  ['a', 'b', 'C', 'D', 'E', 'f', 'g']
  >>> # now remove them
  ... letters[2:5]=[]
  >>> letters
  ['a', 'b', 'f', 'g']
  >>> # clear the list by replacing all the elements with an empty list
  ... letters[:]=[]
  >>> letters
  []
\end{minted}

内置函数 len() 同样适用于列表:
\begin{minted}{python}
  >>> letters=['a','b','c','d']
  >>> len(letters)
  4
\end{minted}

允许嵌套列表(创建一个包含洽谈列表的列表),例如:
\begin{minted}{python}
>>> a=['a','b','c']
>>> n=[1,2,3]
>>> x=[a,n]
>>> x
[['a', 'b', 'c'], [1, 2, 3]]
>>> x[0][1]
'b'
\end{minted}

\subsection{编程的第一步}
当然,我们可以使用Python完成比二加二更复杂的任务。例如,我们可以写一个生成斐波那契子程序的程序,如下所示:
\begin{minted}{python}
>>> # Fibonacci series:
... # the sum of two elements defines the next
... a, b = 0,1
>>> while b<10:
...    print(b)
...    a, b = b, a+b
...
1
1
2
3
5
8
\end{minted}

这个例子介绍了几个新功能。
\begin{itemize}
  \item 第一行包括了一个多重赋值:变量a 和b同时获得了新的值0和1最后一行又使用了一次。在这个演示中,变量赋值前,右边首先完成计算。右边的
  表达式从左到右计算。
  \item 条件(这里是 b<10)为true时,while循环执行。在Python中,类似于C,任何非零整数都是true;0是false。条件也是可以
  是字符串或列表,实际上可以使任何序列;所有长度不为零的是true,空序列是false。示例中的测试是一个检点的比较。
  标准比较操作符与C相同:< , > , == , <= , >= 和 !=。
  \item 循环体是缩进的:缩进是Python组织语句的方法。Python(还)不提供集成的行编辑功能,所以你要为每一个缩
  进行输入TAB或空格。

  实践中建议你找个文本编辑来录入复杂的Python程序,大多数文本编辑器提供自动缩进。交互式录入复合语句时,必须在最后一个
  空行来标识结束(因为解释器没办法猜测你输入的哪一行是最后一行),需要注意的是同一个语句块中的每一行必须缩进同样数量的空白
  \item 关键字print() 语句输出给定表达式的值。它控制多个表达式和字符串输出为你想要字符串(就像在前面计算器的例子中那样)。

  字符串打印时不用引号包围,每两个子项之间插入空间,所以你可以把格式弄得很漂亮,像这样
  \begin{minted}{python}
    >>> i=256*256
    >>> print('The value of i is',i)
    The value of i is', 65536
  \end{minted}

  一个逗号结尾就可以禁止输出换行:
  \begin{minted}{python}
    >>> a, b=0 ,1
    >>> while b < 1000:
    ...     print(b, end=',')
    1,1,2,3,5,8,13,21,34,55,89,144,233,377,610,987,
  \end{minted}
\end{itemize}

\section{深入Python流程控制}
除了前面介绍的while语句,Python还从其他语言借鉴了一些流程控制功能,并有所改变。
\subsection{if语句}
也许最有名的就是if语句。例如:
\begin{minted}{python}
  >>> x = int(input("Please enter an integer: "))
  Please enter an integer: 42
  >>> if x < 0:
  ...     x = 0
  ...     print('Negative changed to zero')
  ... elif x==0:
  ...     print('Zero')
  ... elif x==1:
  ...     print('Single')
  ... else:
  ...     print('More')
  ...
  More
\end{minted}
可能会有零到多个elif部分,else是可选的。关键字`elif'是`else if'的缩写,这个可以有效地避免
过深的缩进。if ... elif ... elif ...序列用于代替其他语言中的 swith 或 case 语句。
\subsection{for语句}
Python中的for语句和C或Pascal中的略有不同。通常的循环可能会依据一个等差数值步进过程(如Pascal)
,或由用户来定义迭代步骤和终止条件(如C),Python的for语句依据任意序列(链表或字符串)中的子项
,按它们在序列中的顺序来进行迭代。例如(没有暗指):
\begin{minted}{python}
>>> # Measure some strings:
... words = ['cat', 'window', 'defenestrate']
>>> for w in words:
...     print(w, len(w))
...
cat 3
window 6
defenestrate 12
\end{minted}
在迭代过程中修改序列不安全(只有在使用链表这样的可变序列时才会有这样的情况)。
如果你想要修改你的迭代的序列(例如,复制选择项),你可以迭代它的副本。使用切割标识就可以很方便
的做到这一点:
\begin{minted}{python}
>>> for w in words[:]:
...     if len(w) > 6:
...             words.insert(0, w)
...
>>> words
['defenestrate', 'cat', 'window', 'defenestrate']
\end{minted}
\subsection{range()函数}
如果你需要一个数值序列,内置函数range()会很方便,它生成一个等差级数链表:
\begin{minted}{python}
>>> for i in range(5):
...     print(i)
...
0
1
2
3
4
\end{minted}
range(10)生成了一个包含10个值的链表,他用链表的索引值填充了这个长度为10的列表,
所生成的链表中不包括范围中的结束值。也可以让range()操作从另一个数值开始,或者可以指定一个不同
的步进值(甚至是负数,有时这也被称为步长)
\begin{minted}{python}
range(5, 10)
  5 through 9
range(0, 10, 3)
  0, 3, 6, 9
range(-10, -100, -30)
  -10, -40, -70
\end{minted}
需要迭代链表索引的话,如下所示结合使用用 range() 和 len()
\begin{minted}{python}
>>> a = ['Mary', 'had', 'a', 'little', 'lamb']
>>> for i in range(len(a)):
...     print(i, a[i])
...
0 Mary
1 had
2 a
3 little
4 lamb
\end{minted}
不过,这种场合可以方便的使用 enumerate(),请参见 循环技巧。

如果你只是打印一个序列的话会发生奇怪的事情:
\begin{minted}{python}
>>> print(range(10))
range(0, 10)
\end{minted}
在不同方面range()函数返回的对象表现为它是一个列表,但事实上它并不是。当你迭代它时,
它是一个能够像期望的序列返回连续项的对象;但为了节省空间,他并不真正构造列表。

我们称此类对象是可迭代的,即适合作为那些期望从某些东西中获得连续项直到结束的函数
或结构的一个目标(参数)。我们已经见过的for语句就是这样一个迭代器。list()函数是
另外一个(迭代器),他从可迭代(对象)中创建列表:
\begin{minted}{python}
>>> list(range(5))
[0, 1, 2, 3, 4]
\end{minted}
稍后我们会看到更多返回可迭代(对象)和以可迭代(对象)作为参数的函数。

\subsection{break 和 continue 语句,以及循环中的else子句}
break语句和C中的类似,用于跳出最近的一级for或while循环。

循环可以有一个else子句;它在循环迭代完整个列表(对于for)或执行条件为false(对于while)时
执行,但循环被break中止的情况不会执行。以下搜索素数的示例程序演示了这个子句:
\begin{minted}{python}
>>> for n in range(2, 10):
...     for x in range(2, n):
...             if n % x == 0:
...                     print(n, 'equals', x, '*', n//x)
...                     break
...     else:
...         # loop fell through without finding s factor
...             print(n, 'is a prime number')
...
2 is a prime number
3 is a prime number
4 equals 2 * 2
5 is a prime number
6 equals 2 * 3
7 is a prime number
8 equals 2 * 4
9 equals 3 * 3
\end{minted}
(Yes,这是正确的代码。看仔细:else 语句是属于 for 循环之中,不是 if 语句。)

与循环一起使用时,else子句与try语句的else子句比与if语句的具有更多的共同特点:try 语句的 else
子句在未出现异常时运行,循环的else子句在未出现break时运行。更多关于 try 语句和异常的内容,请参见异常处理。

continue 语句是从C中借鉴来的,它表示循环继续执行下一次迭代:
\begin{minted}{python}
  >>> for num in range(2, 10):
  ...     if num % 2 ==0:
  ...             print("Found an even number", num)
  ...             continue
  ...     print("Found a number", num)
  ...
  Found an even number 2
  Found a number 3
  Found an even number 4
  Found a number 5
  Found an even number 6
  Found a number 7
  Found an even number 8
  Found a number 9
\end{minted}

\subsection{pass语句}
pass语句什么也不做。它用于那些语法上必须要有什么语句,但程序什么也不做的场合,例如:
\begin{minted}{python}
  >>> while True:
  ...     pass # Busy-wait for keyboard interrupt (Ctrl+C)
  ...
\end{minted}
者通常用于创建最小结构的类:
\begin{minted}{python}
>>> class MyEmptyClass:
...     pass
...
\end{minted}
另一方面,pass可以在创建新代码时用来做函数或控制体的占位符。可以让你在更抽象的级别上思考。
pass可以默默的被忽视:
\begin{minted}{python}
>>> def initlog(*args):
...     pass  # Remember to implement this!
...
\end{minted}

\subsection{定义函数}
我们可以创建一个用来生成指定边界的斐波那契数列的函数:
\begin{minted}{python}
>>> def fib(n):  # write Fibonacci series up to n
...     """Print a Fibonacci series up to n."""
...     a, b = 0, 1
...     while a < n:
...             print(a, end=' ')
...             a, b = b, a+b
...     print()
...
>>> # Now call the function we just defined:
... fib(2000)
0 1 1 2 3 5 8 13 21 34 55 89 144 233 377 610 987 1597
\end{minted}

关键字 def 引入了一个函数定义。在其后必须跟有函数名和包括形式参数的圆括号。函数体语句从下一行
开始,必须是缩进的。

函数体第一行语句可以是可选的字符串文本,这个字符串是函数的文档字符串,或者称为 docstring。(更多关于 docstring 的
信息请参考文档字符串)有些工具通过 docstrings 自动生成在线的或可打印的文档,或者让用户通过代码交互浏览;
在你的代码中包含 docstrings 是一个好的实践,让它成为习惯吧。

函数调用会为函数局部变量生成一个新的符号表。确切的说,所有函数中的变量赋值都是将值存储在局部符号表。变量引用首先在局部符号表中
查找,然后是包含函数的局部符号表,最后是内置名字表。因此,全局变量不能在函数中直接赋值(除非用 global语句命名),尽管他们可以被引用。

函数引用的实际参数在函数调用时引入局部符号表,因此,实参总是传值调用(这里的值总是一个对象引用,而不是该对象的值)。
一个函数被另一个函数调用时,一个新的局部符号表在调用过程中被创建。

一个函数定义会在当前符号表内引入函数名。函数名指代的值(即函数体)有一个被Python 解释器认定为用户自定义函数的类型。这个值可以
赋予其他的名字(即变量名),然后它也可以被当做函数使用。这可以作为通用的重命名机制:
\begin{minted}{python}
>>> fib
<function fib at 0x102aa29d8>
>>> f=fib
>>> f(100)
0 1 1 2 3 5 8 13 21 34 55 89
\end{minted}
如果你使用过其他语言,你可能会反对说:fib不是一个函数,而是一个方法因为它并不返回任何值。事实上,没有return语句的函数确实会返回一个值,
虽然是一个相当令人厌烦的值(指None)。这个值被称为None(这是一个内建名称)。如果None值是唯一被书写的值,那么在写的时候通常会被解释器
忽略(即不输出任何内容)。如果你确实想看到这个值的输出内容,请使用print()函数:
\begin{minted}{python}
>>> fib(0)
>>> print(fib(0))
None
\end{minted}

定义一个返回斐波那契数列数字列表的函数,而不是打印它,是很简单的:
\begin{minted}{python}
>>> def fib2(n): #return Fibonacci series up to n
...     """Return a list containing the Fibonacci series up to n."""
...     result = []
...     a, b = 0, 1
...     while a < n:
...             result.append(a)  # see below
...             a, b = b, a+b
...     return result
...
>>> f100 = fib2(100) # call it
>>> f100             # write the reslut
[0, 1, 1, 2, 3, 5, 8, 13, 21, 34, 55, 89]
\end{minted}

和以前一样,这个例子演示了一些新的Python功能:
\begin{itemize}
  \item return语句从函数中返回一个值,不带表达式的return返回None。过程结束后也会返回None。
  \item 语句result.append(b)称为链表对象 result的一个方法。方法是一个``属于''某个对象的函数,它被命名为obj.methodename,
  这里的obj是某个对象(可能是一个表达式),methodename是某个在该对象类型定义中的方法的命名。

  不同类型定义不同的方法。不同类型可能有同样名字的方法,但不会混淆。(当你定义自己的对象类型和方法时,可能会出现这种情况,
  class的定义方法详见类)。示例中演示的 append()方法由链表对象定义,它向链表中加入一个新元素。在示例中他等同于
  result =result +[a],不过效率更高。
\end{itemize}

\subsection{深入Python函数定义}
在Python中,你也可以定义包含若干参数的函数。这里有三种可用的形式,也可以混合使用。
\subsubsection{默认参数值}
最常用的一种形式是为一个或多个参数指定默认值。这会创建一个可以使用比定义时允许的参数更少的参数调用的函数,例如:
\begin{minted}{python}
>>> def ask_ok(prompt, retries=4, complaint='Yes or no, please!'):
...     while True:
...             ok = input(prompt)
...             if ok in ('y', 'ye', 'yes'):
...                     return Ture
...             if ok in ('n', 'no', 'nop', 'nope'):
...                     return False
...             retries = retries - 1
...             if retries < 0:
...                     raise OSError('uncooperat user')
...             print(complaint)
\end{minted}

这个函数可以通过几种不同的方式调用:
\begin{itemize}
  \item 只给出必要的参数:ask\_ok $($`Do you really want to quit?'$)$
  \item 给出一个可选的参数:

  ask\_ok$($`Ok to overwrite the file?', 2$)$
  \item 或者给出所有的参数:

  ask\_ok$($`OK to overwrite the file?', 2, `Come only yes or no!'$)$
\end{itemize}

这个例子还介绍了 in 关键字。它测定序列中是否包含某个确定的值。

默认值在函数定义作用域被解析,如下所示:
\begin{minted}{python}
>>> i = 5
>>> def f(arg = i):
...     print(arg)
...
>>> i = 6
>>> f()
5
\end{minted}

重要警告:默认值只被赋值一次。这使得当默认值是可变对象时会有所不同,比如列表、字典或者大多数类的实例。
例如,下面的函数在后续调用过程中会累积(前面)传给它的参数:
\begin{minted}{python}
>>> def f(a, L=[]):
...     L.append(a)
...     return L
...
>>> print(f(1))
[1]
>>> print(f(2))
[1, 2]
>>> print(f(3))
[1, 2, 3]
\end{minted}

如果你不想让默认值在后续调用中累积,你可以像下面一样定义函数:
\begin{minted}{python}
>>> def f(a, L=None):
...     if L is None:
...             L = []
...     L.append (a)
...     return L
\end{minted}

\subsubsection{关键字参数}
函数可以通过关键字参数的形式来调用,形如 keyword = value。例如,以下的函数:
\begin{minted}{python}
>>> def parrot(voltage, state='a stiff', action='voom', type='Norwegian Blue'):
...     print("-- This parrot wouldn't", action, end=' ')
...     print("if you put", voltage, "volts through it.")
...     print("-- Lovely plumage, the", type)
...     print("-- It's", state, "!")
\end{minted}

接受一个必选参数(voltage)以及三个可选参数(state, action和type)。可以用以下的任一方法调用:
\begin{minted}{python}
>>> parrot(1000)                      # 1 positional argument
-- This parrot wouldn't voom if you put 1000 volts through it.
-- Lovely plumage, the Norwegian Blue
-- It's a stiff !
>>> parrot(voltage=1000)              # 1 keyword argument
-- This parrot wouldn't voom if you put 1000 volts through it.
-- Lovely plumage, the Norwegian Blue
-- It's a stiff !
>>> parrot(voltage=1000000, action='VOOOOOM')  # 2 keyword arguments
-- This parrot wouldn't VOOOOOM if you put 1000000 volts through it.
-- Lovely plumage, the Norwegian Blue
-- It's a stiff !
>>> parrot(action='VOOOOOM',voltage=1000000)   # 2 keyword arguments
-- This parrot wouldn't VOOOOOM if you put 1000000 volts through it.
-- Lovely plumage, the Norwegian Blue
-- It's a stiff !
>>> parrot('a thousand', state='pushing up the daisies')  # 1 positional, 1 keyword
-- This parrot wouldn't voom if you put a thousand volts through it.
-- Lovely plumage, the Norwegian Blue
-- It's pushing up the daisies !
\end{minted}

不过以下几种调用是无效的:
\begin{minted}{python}
>>> parrot()                     # requried argument missing
>>> parrot(voltage=5.0, 'dead')  # non-keyword argument after a keyword argument
>>> parrot(110, voltage=220)     # duplicate value for the same argument
>>> parrot(actor='John Cleese')  # unknow keyword argument
\end{minted}

在函数调用中,关键字的参数必须跟随在位置参数的后面。传递的所有关键字参数必须与函数接受的某个参数相匹配
(例如 actor 不是 parrot 函数的有效参数),它们的顺序并不重要。这也包括非可选参数(例如 parrot(voltage=100)也是有效的)。
任何参数都不可以多次赋值。下面的示例由于这种限制将失败:
\begin{minted}{python}
>>> def function(a):
...     pass
...
>>> function(0, a=0)
Traceback (most recent call last):
  File "<stdin>", line 1, in <module>
TypeError: function() got multiple values for argument 'a'
\end{minted}

引入一个形如 $**$name 的参数时,它接收一个字典(参见Mapping Types-dict),该字典包含了所有未出现在形式参数列表中的关键字参数。
这里可能还会组合使用一个形如 $*$name (下一小节详细介绍)的形式参数,它接收一个元组(下一节中会详细介绍),包含了所有没有出现在形式参数
列表中的参数值( $*$name必须在 $**$name之前出现)。例如,我们这样定义一个函数:
\begin{minted}{python}
>>> def cheeseshop(kind, *arguments, **keywords):
...     print("-- Do you have any", kind, "?")
...     print("-- I'm sorry. we're all out of", kind)
...     for arg in arguments:
...             print(arg)
...     print("-" * 40)
...     keys = sorted(keywords.keys())
...     for kw in keys:
...             print(kw, ":", keywords[kw])
\end{minted}

它可以像这样调用:
\begin{minted}{python}
>>> cheeseshop("Limburger", "It's very runny, sir.",
...             "It's really very, VERY runny, sir.",
...             shopkeeper="Michael Palin."
...             client="John Cleese",
...             sketch="Cheese Shop Sketch")
\end{minted}

当然它会按如下内容打印:
\begin{minted}{python}
-- Do you have any Limburger ?
-- I'm sorry, we're all out of Limburger
It's very runny, sir.
It's really very, VERY runny, sir.
----------------------------------------
client : John Cleese
shopkeeper : Michael Palin
sketch : Cheese Shop Sketch
\end{minted}

注意在打印关键字参数之前,通过对关键字字典 keys() 方法的结果进行排序,生成了关键字参数名的列表;
如果不这样做,打印出来的参数顺序是未定义的。

\subsubsection{可变参数列表}
最后,一个最不常用的选择是可以让函数调用可变个数的参数。这些参数被包装进一个元祖(参见 元祖和序列)。
在这个可变个数的参数之前,可以有零到多个普通的参数:
\begin{minted}{python}
>>> def concat(*args, sep="/"):
...     return sep.join(args)
...
>>> concat("earth", "mars", "venus")
'earth/mars/venus'
>>> concat("earth", "mars", "venus", sep=".")
'earth.mars.venus'
\end{minted}
\subsubsection{参数列表的拆分}
另一种相反的情况:当你要传递的参数已经是一个列表,但要调用的函数却接受分开一个个的参数值。
这时候你要把已有的列表拆开来。例如内建函数 range()需要要独立的 start, stop参数。
你可以在调用函数时加一个$*$操作符来自动把参数列表拆开:
\begin{minted}{python}
>>> list(range(3, 6))   # normal call with separate arguments
[3, 4, 5]
>>> args = [3, 6]
>>> list(range(*args))  # call with arguments unpacked from a list
[3, 4, 5]
\end{minted}

以同样的方法,可以使用$**$操作符拆分关键字参数为字典:
\begin{minted}{python}
>>> def parrot(voltage, state='a stiff', action='voom'):
...     print("-- This parrto wouldn't", action, end=' ')
...     print("if you put", voltage, "volts through it.", end=' ')
...     print("E's", state, "!")
...
>>> d = {"voltage": "four million", "state": "bleedin' demised", "action": "VOOM"}
>>> parrot(**d)
-- This parrto wouldn't VOOM if you put four million volts through it. E's bleedin' demised !
\end{minted}

\subsubsection{Lambda形式}
出于实际需要,有几种通常在函数式编程语言例如Lisp中出现的功能加入到了Python。通过lambda关键字,
可以创建短小的匿名函数。这里有一个函数返回它的两个参数的和:lambda a, b: a+b。Lambda形式可以用于任何
需要的函数对象。出于语法限制,它们只能有一个单独的表达式。
\begin{minted}{python}
>>> def make_incrementor(n):
...     return lambda x: x + n
...
>>> f = make_incrementor(42)
>>> f(0)
42
>>> f(1)
43
\end{minted}

上面的示例使用lambda表达式返回一个函数。另一个用途是将一个小函数作为参数传递:
\begin{minted}{python}
>>> pairs = [(1, 'one'), (2, 'two'), (3, 'three'), (4, 'four')]
>>> pairs.sort(key=lambda pair: pair[1])
>>> pairs
[(4, 'four'), (1, 'one'), (3, 'three'), (2, 'two')]
\end{minted}

\subsubsection{文档字符串}
这里介绍的文档字符串的概念和格式。

第一行应该是关于对象用途的简介。简短起见,不用明确的陈述对象名或类型,
因为它们可以从别的途径了解到(除非这个名字碰巧就是描述这个函数操作的动词)。这一行应该
以大写字母开头,以句号结尾。

如果文档字符串有多行,第二行应该空出来,与接下俩的详细描述明确分隔。接下来的文档婴孩有一
或多段描述对象的调用约定、边界效应等。

Python的解释器不会从多行的文档字符串中取除缩进,所以必要的时候应当自己清除缩进。这符合通常的习惯。
第一行之后的第一个非空行决定了整个文档的缩进格式。(我们不用第一行是因为它通常紧靠着起始的引号,缩进
的话,所有的留白都应该清除掉。留白的长度应当等于扩展制表符的宽度(通常是8个空格)。

以下是一个多行文档字符串的示例:
\begin{minted}{python}
>>> def my_function():
...     """Do nothing, but document it.
...
...     No, really, it doesn't do anything.
...     """
...     pass
...
>>> print(my_function.__doc__)
Do nothing, but document it.

	No, really, it doesn't do anything.
\end{minted}

\section{数据结构}
\subsection{关于列表的更多的内容}
Python的列表数据类型包含更多的方法。这里是所有的列表对象方法:
\begin{itemize}
  \item list.append(x) 把一个元素添加到列表的结尾,相当于 a[len(a):] = [x]
  \item list.extend(L) 将一个给定列表中的所有元素都添加到另一个列表中,相当于 a[len(a):]=L
  \item list.insert(i,x) 在指定位置插入一个元素。第一个参数是准备插入到其前面的那个元素的索引,
  例如 a.insert(0, x)会插入到整个列表之前,而 a.insert(len(a), x)相当于 a.append(x)。
  \item list.remove(x) 删除列表中值为x的第一个元素。如果没有这样的元素,就会返回一个错误。
  \item list.pop([i]) 从列表的指定位置删除元素,并将其返回。如果没有指定索引,a.pop()返回最后一个元素。
  元素随机从列表中被删除(方法中i两边的方括号表示这个参数是可选的,而不是要求你的输入一对方括号,你会经常
  在Python库参考手册中遇到这样的标记)。
  \item list.clear() 从列表中删除所有元素。相当于 del a[:]。
  \item list.count(x) 返回x在列表中出现的次数
  \item list.index(x) 返回列表中第一个值为x的元素的索引。如果没有匹配的元素就会返回一个错误。
  \item list.sort() 对类表中的元素就地进行排列。
  \item list.reverse() 就地倒排列表中的元素。
  \item list.copy() 返回列表的一个浅拷贝。等同于 a[:]。
\end{itemize}

下面这个示例演示了列表的大部分方法:
\begin{minted}{python}
>>> a = [66.25, 333, 333, 1, 1234.5]
>>> print(a.count(333), a.count(66.25), a.count('x'))
2 1 0
>>> a.insert(2, -1)
>>> a.append(333)
>>> a
[66.25, 333, -1, 333, 1, 1234.5, 333]
>>> a.index(333)
1
>>> a.remove(333)
>>> a
[66.25, -1, 333, 1, 1234.5, 333]
>>> a.reverse()
>>> a
[333, 1234.5, 1, 333, -1, 66.25]
>>> a.sort()
>>> a
[-1, 1, 66.25, 333, 333, 1234.5]
>>> a.pop()
1234.5
>>> a
[-1, 1, 66.25, 333, 333]
\end{minted}

也许大家会发现像insert, remove 或者sort这些修改列表的方法没有打印返回值——它们
返回None。在python中对所有可变的数据类型这是统一的设计原则。
\subsubsection{把列表当作堆栈使用}
列表方法使得列表可以很方便的作为一个堆栈来使用,堆栈作为特定的数据结构,最先进入的元素
最后一个被释放(后进先出)。用 append() 方法可以把一个元素添加到堆栈顶。用不指定索引的
 pop() 方法可以把一个元素从堆栈顶释放出来。例如:
 \begin{minted}{python}
 >>> stack = [3, 4, 5]
>>> stack.append(6)
>>> stack.append(7)
>>> stack
[3, 4, 5, 6, 7]
>>> stack.pop()
7
>>> stack
[3, 4, 5, 6]
>>> stack.pop()
6
>>> stack.pop()
5
>>> stack
[3, 4]
\end{minted}
\subsubsection{把列表当做队列使用}
你也可以把列表当做队列使用,队列作为特定的数据结构,最先进入的元素最先释放(先进先出)。
不过,列表这样用效率不高。相对来说从列表末尾添加和弹出很快;在头部插入和弹出很慢(因为,
为了一个元素,要移动整个列表中的所有元素)。

要实现队列,使用 collections.deque,它为在首尾两端快速插入和删除而设计。例如:
\begin{minted}{python}
>>> from collections import deque
>>> queue = deque(["Eric", "John", "Michael"])
>>> queue.append("Terry")               # Terry arrives
>>> queue.append("Graham")              # Graham arrives
>>> queue.popleft()                     # The first to arrive now leaves
'Eric'
>>> queue.popleft()                     # The second to arrive now leaves
'John'
>>> queue                               # Remaining queue in order of arrival
deque(['Michael', 'Terry', 'Graham'])
\end{minted}
\subsubsection{列表推导式}
列表推导式为从序列中创建列表提供了一个简单的方法。普通的应用程式通过将一些操作应用于
序列的每个成员并通过返回的元素创建列表,或者通过满足特定条件的元素创建子序列。
\begin{minted}{python}
>>> squares = []
>>> for x in range(10):
...     squares.append(x**2)
...
>>> squares
[0, 1, 4, 9, 16, 25, 36, 49, 64, 81]
\end{minted}

注意这个for循环中的被创建(或被重写)的名为x的变量在循环完毕后依然存在。使用如下方法,我们可以计算
squares的值而不会产生任何的副作用:
\begin{minted}{python}
>>> squares = list(map(lambda x: x**2, range(10)))
\end{minted}

或者等价于:
\begin{minted}{python}
squares = [x**2 for x in range(10)]
\end{minted}

上面这个方法更加简明且易读。

列表推导式由包含一个表达式的括号组成,表达式后面跟随一个for子句,之后可以有零个或多个for或if子句。
结果是一个列表,由表达式依据其后面的for和if子句上下文计算而来的结果构成。

例如,如下的列表推导式结合两个列表的元素,如果两个元素之间不相等的话:
\begin{minted}{python}
>>> [(x, y) for x in [1,2,3] for y in [3,1,4] if x !=y]
[(1, 3), (1, 4), (2, 3), (2, 1), (2, 4), (3, 1), (3, 4)]
\end{minted}

等同于:
\begin{minted}{python}
>>> combs = []
>>> for x in [1,2,3]:
...     for y in [3,1,4]:
...             if x != y:
...                     combs.append((x,y))
...
>>> combs
[(1, 3), (1, 4), (2, 3), (2, 1), (2, 4), (3, 1), (3, 4)]
\end{minted}

值得注意的是在上面两个方法中的for和if语句的顺序。

如果想要得到一个元组(例如,上面例子中的(x,y)),必须要加上括号:
\begin{minted}{python}
>>> vec = [-4, -2, 0, 2, 4]
>>> # create a new list with the values doubled
>>>[x*2 for x in vec]
[-8, -4, 0, 4, 8]
>>> # filter the list to exclude negative numbers
>>> [x for x in vec if x >= 0]
[0, 2, 4]
>>> # apply a function to all the elements
>>> [abs(x) for x in vec]
[4, 2, 0, 2, 4]
>>> # call a method on each elements
>>> freshfruit = [' banana', ' loganberry', 'passion fruit ']
>>> [weapon.strip() for weapon in freshfruit]
['banana', 'loganberry', 'passion fruit']
>>> # create a list of 2-tuples like (number, square)
>>> [(x, x**2) for x in range(6)]
[(0, 0), (1, 1), (2, 4), (3, 9), (4, 16), (5, 25)]
>>> # the tuple must be parenthesized, otherwise an error is raised
>>> [x, x**2 for x in range(6)]
  File "<stdin>", line 1
    [x, x**2 for x in range(6)]
               ^
SyntaxError: invalid syntax
>>> # flatten a list using a listcomp with two 'for'
>>> vec = [[1,2,3], [4,5,6], [7,8,9]]
>>> [num for elem in vec for num in elem]
[1, 2, 3, 4, 5, 6, 7, 8, 9]
\end{minted}

列表推导式可使用复杂的表达式和嵌套函数:
\begin{minted}{python}
>>> from math import pi
>>> [str(round(pi, i)) for i in range(1, 6)]
['3.1', '3.14', '3.142', '3.1416', '3.14159']
\end{minted}
\subsubsection{嵌套的列表推导式}
列表解析中的第一个表达式可以是任何表达式,包括列表解析。

考虑下面由三个长度为4的列表组成3$\times$4矩阵:
\begin{minted}{python}
>>> matrix = [
...     [1, 2, 3, 4],
...     [5, 6, 7, 8],
...     [9, 10, 11, 12],
... ]
\end{minted}

现在如果你想交换行和列,可以用嵌套的列表推导式:
\begin{minted}{python}
>>> [[row[i] for row in matrix] for i in range(4)]
[[1, 5, 9], [2, 6, 10], [3, 7, 11], [4, 8, 12]]
\end{minted}

像前面看到的,嵌套的列表推导式是对for后面的内容进行求值,所以上例就等价于:
\begin{minted}{python}
>>> transposed = []
>>> for i in range(4):
...     # the following 3 lines implement the nested listcomp
...     transposed_row = []
...     for row in matrix:
...             transposed_row.append(row[i])
...     transposed.append(transposed_row)
...
>>> tramsposed
[[1, 5, 9], [2, 6, 10], [3, 7, 11], [4, 8, 12]]
\end{minted}

在实际中,你应该更喜欢使用内置函数组成复杂流程语句。对这种情况 zip()函数将会做的更好:
\begin{minted}{python}
>>> list(zip(*matrix))
[(1, 5, 9), (2, 6, 10), (3, 7, 11), (4, 8, 12)]
\end{minted}

更多关于本行中使用的星号的说明,参考 参数列表的拆分。
\subsection{del语句}
有个方法可以从类表中按给定的索引而不是值来删除一个子项:del语句。它不同于有返回值的
pop()方法。语句del还可以从列表中删除切片或清空整个列表(我们以前介绍过一个方法是将空列表
赋值给列表的切片)。例如:
\begin{minted}{python}
>>> a = [-1, 1, 66.25, 333, 333, 1234.5]
>>> del a[0]
>>> a
[1, 66.25, 333, 333, 1234.5]
>>> del a[2:4]
>>> a
[1, 66.25, 1234.5]
>>> del a[:]
>>> a
[]
\end{minted}

del也可以删除整个变量:
\begin{minted}{python}
>>> del a
>>> a
\end{minted}

此后再引用命名a会引发错误(直到另一个值赋给它为止)。我们在后面的内容中可以看到del的其它用法。
\subsection{元组和序列}
我们知道列表和字符串有很多通用的属性,例如索引和切割操作。它们是序列类型(参见Sequence Types -
list,tuple,range) 中的两种。因为Python是一个在不停进化的语言,也可能会加入其它的序列类型,这里
介绍另一种标准序列类型:元组。

一个元组由数个逗号分隔的值组成,例如:
\begin{minted}{python}
>>> t = 12345, 54321, 'hello!'
>>> t[0]
12345
>>> t
(12345, 54321, 'hello!')
>>> # Tuples may be nested:
... u = t, (1, 2, 3, 4, 5)
>>> u
((12345, 54321, 'hello!'), (1, 2, 3, 4, 5))
>>> # Tuples are immutable:
... t[0] = 88888
Traceback (most recent call last):
  File "<stdin>", line 2, in <module>
TypeError: 'tuple' object does not support item assignment
>>> # but they can contain mutable objects:
... v = ([1, 2, 3], [3, 2, 1])
>>> v
([1, 2, 3], [3, 2, 1])
\end{minted}
正如你所见,元组在输出时总是有括号的,以便于正确表达嵌套结构。在输入时
可以有或没有括号,不过经常括号都是必须的(如果元组是一个更大的表达式的一部分)。
不能给元组的一个独立的元素赋值(尽管你可以通过联接和切割来模拟)。还可以创建包含可变
对象的元组,例如列表。

虽然元组和列表很类似,它们经常被用来在不同的情况和不同的用途。元组有很多用途。例如(x,y)
坐标对,数据库中的员工记录等等。元组就像字符串,不可变的。通常包含不同种类的元素并通过分拆(参阅
本节后面的内容)或索引访问(如果是namedtuples,甚至可以通过属性)。列表是可变的,它们的元素通常是相同类型的并通过迭代访问。

一个特殊的问题是构造包含零个或一个元素的元组:为了适应这种情况,语法上有一些额外的改变。一对空的括号可以创建空元组;
要创建一个单元素元组可以在值后面跟一个逗号(在括号中放入一个单值不够明确)。丑陋,但是有效。例如:
\begin{minted}{python}
>>> empty = ()
>>> singleton = 'hello',     # <-- note trailing comma
>>> len(empty)
0
>>> len(singleton)
1
>>> singleton
('hello',)
\end{minted}

语句 t = 12345, 54321, `hello!' 是元组封装(tuple packing)的一个例子:值12345,54321和`hello'
被封装进元组。其逆操作可能是这样:
\begin{minted}{python}
>>> x, y, z = t
\end{minted}

这个调用等号右边可以是任何线性序列,称之为序列拆分非常恰当。序列拆封要企业左侧的变量数目与序列的元素个数相同。要
注意的是可变参数(multiple assignment)其实只是元组封装的序列拆封的一个结合。

\subsection{集合}
Python还包含了一个数据类型--set(集合)。集合是一个无序不重复元素的集。基本功能包括关系测试和消除重复元素。集合对象还支持
union(联合),intersection(交),difference(差)和sysmmetric difference(对称差集)等数学运算。

大括号或 set() 函数可以用来创建集合。注意:想要创建空集合,你必须要使用 set() 而不是 {}。
后者用于创建空字典。以下是一个简单的演示:
\begin{minted}{python}
>>> basket = {'apple', 'orange', 'apple', 'pear', 'orange', 'banana'}
>>> print(basket)                     # show that duplicates have been removed
{'pear', 'apple', 'orange', 'banana'}
>>> 'orange' in basket                 # fast membership testing
True
>>> 'crabgrass' in basket
False
>>> a = set('abracadabra')
>>> b = set('alacazam')
>>> a                      # unique letters in a
{'r', 'a', 'd', 'c', 'b'}
>>> a - b                  # letters in a but not in b
{'d', 'r', 'b'}
>>> a | b                  # letters in either a or b
{'r', 'a', 'l', 'd', 'c', 'z', 'm', 'b'}
>>> a & b                  # letters in both a and b
{'a', 'c'}
>>> a ^ b                  # letters in a or b but not both
{'r', 'l', 'z', 'd', 'm', 'b'}
\end{minted}

类似列表推导式,这里有一种集合推导式语法:
\begin{minted}{python}
>>> a = {x for x in 'abracadabra' if x not in 'abc'}
>>> a
{'d', 'r'}
\end{minted}

\subsection{字典}
另一个非常有用的Python内建数据类型 是字典(参见Mapping Types - dict)。字典在
某些语言中可能称为 联合内存(associative memories)或联合数组(associative arrays)。
序列是以连续的整数为索引,与此不同的是,字典以关键字为索引,关键字可以是任意不可变类型,通常用字符串或数值。
如果元组中只包含字符串和数字,他可以作为关键字,如果它直接或间接的包含了可变对象,就不能当做
关键字。不能用列表做关键字,因为列表可以用索引、切割或者appen()和extend()等方法改变。

理解字典的最佳方式是把它看做无序的键:值对(key:value对)集合,键必须是互不相同的(在同一个字典之内)。
一对大括号创建一个空的字典: {} 。初始化列表时,在大括号内放置一组逗号分隔的键:值对,这也是字典输出的方式。

字典的主要操作是依据键来存储和析取值。也可以用 del 来删除键:值对(key:value)。如果你用一个已经存在的关键字
存储值,以前为该关键字分配的值就会被遗忘。试图从一个不存在的键中取值会导致错误。

对一个字典执行 list(d.keys()) 将返回一个字典中所有关键字组成的无序列表(如果你想要排序,只需使用 sorted(d.key())
使用 in 关键字(指Python语法)可以检查字典中是否存在某个关键字(指字典)。

字典示例:
\begin{minted}{python}
>>> tel = {'jack': 4098, 'sape': 4139}
>>> tel['guido'] = 4127
>>> tel
{'jack': 4098, 'sape': 4139, 'guido': 4127}
>>> tel['jack']
4098
>>> del tel['sape']
>>> tel['irv'] = 4127
>>> tel
{'jack': 4098, 'guido': 4127, 'irv': 4127}
>>> 'guido' in tel
True
>>> 'jack' not in tel
False
\end{minted}

dict()构造函数可以直接从key-value对中创建字典:
\begin{minted}{python}
>>> dict([('sape', 4139), ('guido', 4127), ('jack', 4098)])
{'sape': 4139, 'guido': 4127, 'jack': 4098}
\end{minted}

此外,字典推导可以从任意的键值表达式中创建字典:
\begin{minted}{python}
>>> {x: x**2 for x in (2, 4, 6)}
{2: 4, 4: 16, 6: 36}
\end{minted}

如果关键字都是简单的字符串,有时通过关键字参数指定 key-value 对更为方便:
\begin{minted}{python}
>>> dict(sape=4139, guido=4127, jack=4098)
{'sape': 4139, 'guido': 4127, 'jack': 4098}
\end{minted}

\subsection{循环技巧}
在字典中循环时,关键字和对应的值可以使用 items() 方法同时解读出来:
\begin{minted}{python}
>>> knights = {'gallahad': 'the pure', 'robin': 'the brave'}
>>> for k,v in knights.items():
...     print(k, v)
...
gallahad the pure
robin the brave
\end{minted}

在序列中循环时,索引位置和对应值可以使用 enumerate() 函数同时得到:
\begin{minted}{python}
>>> for i, v in enumerate(['tic', 'tac', 'toe']):
...     print(i, v)
...
0 tic
1 tac
2 toe
\end{minted}

同时循环两个或更多的序列,可以使用 zip() 整体打包:
\begin{minted}{python}
>>> questions = ['name', 'quest', 'favorite color']
>>> answers = ['lancelot', 'the holy grail', 'blue']
>>> for q, a in zip(questions, answers):
...     print('What is your {0}? It is {1}.' .format(q, a))
...
What is your name? It is lancelot.
What is your quest? It is the holy grail.
What is your favorite color? It is blue.
\end{minted}

需要逆向循环序列的话,先正向定位序列,然后调用 reversed() 函数:
\begin{minted}{python}
>>> for i in reversed(range(1, 10, 2)):
...     print(i)
...
9
7
5
3
1
\end{minted}

要按排序后的顺序循环序列的话,使用sorted()函数,他不改动原序列,而是生成一个新的已排序的序列:
\begin{minted}{python}
>>> basket = ['apple', 'orange', 'apple', 'pear', 'orange', 'banana']
>>> for f in sorted(set(basket)):
...     print(f)
...
apple
banana
orange
pear
\end{minted}

若要在循环内部修改正在遍历的序列(例如复制某些元素),建议您首先制作副本。在序列上循环不会隐式地创建
副本。切片表示法使用这尤其方便:
\begin{minted}{python}
>>> words = ['cat', 'window', 'defenstrate']
>>> for w in words[:]:      # Loop over a slice copy of the entire list.
...     if len(w) > 6:
...             words.insert(0, w)
...
>>> words
['defenstrate', 'cat', 'window', 'defenstrate']
\end{minted}

\subsection{深入条件控制}
while 和 if 语句中使用的条件不仅可以使用比较,而且可以包含任意的操作。

比较操作符 in 和 not in 审核值是否在一个区间之内。操作符 is 和 is not 比较两个对象时否相同;
这只和诸如列表这样的可变对象有关。所有的比较操作符具有相同的优先级,低于所有的数值操作。

比较操作可以传递。例如 a < b == c 审核是否 a 小于 b 并且 b 等于 c。

比较操作可以通过逻辑操作符 and 和 or 组合,比较的结果可以用 not 来取反义。这些操作符的优先级又低于比较操作符,
在他们之中, not 具有最高的优先级,or 优先级最低,所以A and not B or C 等于(A and (notB)) or C。当然,括号也可以
用于比较表达式。

逻辑操作符 and 和 or 也称为短路操作符:它们的参数从左向右解析,一旦结果可以确定就停止。例如,如果 A 和 C 为真而
B 为假,A and B and C 不会解析C。作用于一个普通的非逻辑值时,短路操作符的返回值通常是最后一个变量。

可以把比较或其它逻辑表达式的返回值赋给一个变量,例如:
\begin{minted}{python}
>>> string1, string2, string3 = '', 'Tronheim', 'Hammer Dance'
>>> non_null = string1 or string2 or string3
>>> non_null
'Tronheim'
\end{minted}

需要注意的是Python与C不同,在表达式内部不能赋值。C程序员经常对此抱怨,






\end{document}
