%-*- coding: UTF-8 -*-
\documentclass[UTF8]{ctexart}
\usepackage{geometry}
\geometry{a4paper,centering,scale=0.8}
\usepackage{graphicx}
\usepackage{amsmath}
\usepackage{textcomp}
\usepackage{amsthm}
\usepackage{amssymb}

\title{\heiti Python 基础篇}
\author{卢婧宇}

\begin{document}
\maketitle
\tableofcontents

\newpage
\section{介绍}
\subsection{Python是什么}
Python是一门高级程序设计语言,它的创作者是Guido van Rossum,在1989年圣诞节的时候编写。
\subsection{Python的logo}
蟒蛇,Python是一个马戏团Monty Python的名字,Guido喜欢看它的演出。
\subsection{Python的特点}
1.面向对象
\begin{itemize}
  \item Python完全支持OOP
  \item 也同时支持面向过程
\end{itemize}

2.免费
\begin{itemize}
  \item Python完全免费
  \item 但同时有强大的支持
\end{itemize}

3.跨平台
\begin{itemize}
  \item 有标准C编写,支持多种平台
  \item 一次编写,处处运行
\end{itemize}

4.功能强大
\begin{itemize}
  \item 多种数据类型,丰富的运算
  \item 动态类型,意味着可以直接使用对象
  \item 垃圾自动回收
  \item 众多的模块工具
\end{itemize}

5.胶水性
\begin{itemize}
  \item 可以轻易的和其他语言编写粘连在一起
\end{itemize}

6.简约不简单
\begin{itemize}
   \item  简单易学,代码质量依然可以保障。
\end{itemize}
\section{Python的linux开发环境}
\subsection{Python的安装}
系统自带,终端输入:python -V,查看版本号
\subsection{Python程序运行}
第一种方式,基于解释器方式,交互模式

第二种方式,基于程序文件方式:编写hello.py代码文件。

\begin{itemize}
\item 执行方式一:命令调用文件 python hello.py

\item 执行方式二:让文件自己执行 chmod +x hello.py  ./hello.py 在第一行加上组织头 \#!/usr/bin/python。不可直接执行,没有执行权限

\item 执行方式三:以模块的方式 python -m hello

\item 执行方式四:在文件开头写 \#!/usr/bin/env python 兼容性更好
\end{itemize}

control + d 或者 exit()退出。
\section{Python基础概述——面向过程}
\subsection{数据类型}
基本数据类型:数值、字符串、布尔、None

复合数据类型:list、tuple、dict
\subsection{对应数据类型运算}
基本运算:算数、关系、逻辑运算
\subsection{流程控制}
顺序、分支、异常处理、循环
\subsection{复合数据类型操作 列表}
基本方法、遍历、切片、解析列表
\subsection{复合数据类型操作 元祖}
遍历、切片、解包
\subsection{数据类型间操作}
Boolean、List、tuple、dict
\subsection{函数}
函数定义、函数使用、变量的作用域、参数、匿名函数
\subsection{关于排列}
列表、字典(无序的)
\subsection{常用的内置函数}
Map、Filter、zip
\subsection{文件操作}
将数据保存起来。文件打开模式、文件的基本读写、文件的指针操作
\subsection{模块包管理}
不同的程序文件可以放置在不同的位置,起不同的名字。
包模块结构、导入本模块外资源的方式




\end{document}
